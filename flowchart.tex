\documentclass[12pt]{amsart}
\begin{document}
\section{Parametrizing The triple  reduced product}
Let
$$X=
\begin{pmatrix}
ia &r=p+iq&u+iv\cr
-p+iq&ib&x+iy\cr
-u+iv&-x+iy&ic\cr
\end{pmatrix}$$
in $\mathfrak{su}(3)$ represent a point in the triple reduced product ${\mathcal E}$.

The $4$-dim variety ${\mathcal M}$ is defined by relating the $9$~real variables
$a$, $b$, $c$, $p$, $q$, $u$, $v$, $x$, $y$ by equations $(5)-(8)$ of the paper
together with the condition $a+b+c=0$.

As described in the paper, we do our calculations in the transversal $v=0$, $y=x$.

We parametrize the transversal using $|r|^2$ and~$c$.
The first goal is use the equations to write $a$, $b$, $p$, $q$, $u$, $x$ in terms of our
chosen parameters.

For a $3\times 3$ matrix~$M$, define $\tau_2(M)$ to be the coefficient of $z$ in
$\det(M-z I)$, that is, the linear term in the characteristic polynomial.
In the expression $-\tau_2(-X-i\mu)+\tau_2(X)$ the quadratic terms cancel so equation~(5)
becomes a linear function of $a$, $b$, $c$.
Solving the simultaneous equations \{(5), $a+b+c=0$\} gives $a$ and $b$ as linear functions
of~$c$, denoted in the code as ``arule'' and ``brule''.

Equations (6), (7), (8) in the paper become $eqn1=0$, $eqn2=0$, $eqn3=0$ in the code.

Replacing $p^2$ by $|r|^2-q^2$ eliminates all occurrences of $p$, $q$ from equations eqn1
and~eqn3 which can now be regarded as linear simultaneous equations in the variables $u^2$
and~$x^2$.
Solving these gives $u^2$ and $x^2$ as functions of $a$, $b$, $c$, $|r|^2$ and replacing
$a$ and $b$ gives ``uurule'' and ``xxrule'' for writing $u^2$ and $x^2$ in terms of our
parameters $|r|^2$ and~$c$.

Replacing $p^2 = |r|^2-q^2$ and $ux = \sqrt{uurule\ xxrule}$ in eqn2 and replacing $a$, $b$
using arule, brule, gives $p+q$ as a function of the parameters $|r|^2$ and ~$c$.
Together with the equation $p^2+q^2=|r|^2$ this determines two curves giving $p$ and $q$
in terms of $|r|^2$ and~$c$, which are interchanged under the involution $p\leftrightarrow q$,
as described in Lemma~2.3.
We pick one solution curve and do our integrals over that, doubling the answer to get
the actual period.


\section{The symplectic form}
The next goal is to find a usable formula for the symplectic form.
In the paper, a function $V(~)$ taking values in the tangent space~$T_X{\mathcal M}$ is
defined.
Formula~(25) for the symplectic form gives a simple formula for the value $\Omega(Z,Y)$,
but only in the case where one of $Y$, $Z$ is given in the form $V(e)$.

We pick a candidate function which we wish to test as being a possible Hamiltonian for some
circle action on~$\mathcal E$.
In this file we are testing the average Gelfand-Cetlin function,
$H:=\bigl(GC(X)+GC(-X-i\mu)\bigr)/2$.


Starting with a random choice of matrices $e_1$, $e_2$ in $\mathfrak{su}(3)$, by comparing
directional derivatives $D_{V(e_1)}H$, $D_{V(e_2)}H$ we find a linear combination~$e$ of
$e_1$ and~$e_2$ such that $D_{V(e)}H=0$.
Because we are in two dimensions, if $H$ is indeed the Hamiltonian for a circle action, 
then $V(e)$ will be a scalar multiple of the Hamiltonian vector field~$X_H$, denoted $\chi$
in the code.

Since we are in two dimensions, the symplectic form is completely determined by knowledge
of $\Omega (Z,Y)$ for any pair of linearly independent tangent vectors $Z$, $Y$.

Writing all the entries of $X$ in terms of $|r|^2$ and $c$ we form the matrix of
partial derivatives with respect to $c$, which gives a tangent vector $T_c$.
Using equation~(25) we evaluate $\hat{\Omega}:=\Omega\bigl(T_c,V(e)\bigr)$.
We are now ready to use linear algebra to evaluate $\Omega(Z,Y)$ for arbitrary $Z$ and~$Y$
by writing $Z$ and~$Y$ in the $T_c, V(e)$ basis.
At this point we could get an expression for $F(r,c)$ to write the symplectic form as
$\Omega = F(r,c)\,d|r|^2\wedge dc$ and go on to calculate the symplectic volume.

Comparing $\hat{\Omega}$ with the value $\Omega(\chi,T_c)$ coming from the definition
of a Hamiltonian now allows us to compute $\chi_H$ as in Section~3.5 of the paper.

We find it convenient to sometimes use the formula for $H$ in terms of $|r|^2$ and~$c$
(which is quartic in $|r|^2$) and switch to parameters $H$ and~$c$.
Writing $\chi_H$ in terms of $H$ and~$c$ we evaluate the $\tau_z$ described in Section~3.5
and obtain the table of values appearing in the paper.

\end{document}
